\section{Introduction}

\subsection{Objet du document}

Ce dossier de spécification a pour objectif de définir les fonctionnalités et exigences attendues par le Projet PATO pour le développement de la partie logiciel d'un robot de nettoyage. Le contenu de ce document portera principalement sur les aspects logiciels de JAR. 

Les fonctionnalités et exigences présentées dans ce document ont été déterminées à la suite de l'étude du Cahier des charges du Projet PATO, Robot de nettoyage. 

Il utilise des schémas et illustrations respectant la norme UML. 

\subsection{Portée du document}

Ce document, décrit dans ce document les caractéristiques et fonctionnalités du Système à l'Étude (SàE), constitué : 

\begin{itemize}
    \item Du logiciel “télécommande”, une application qui permet de télécommander un robot de nettoyage et de voir les métriques de ce dernier. 
    \item Du logiciel “Algorithme de déplacement”, une application  définie le déplacement et le comportement du robot. 
\end{itemize}

\subsection{Copyright}

Le présent document est un document à but pédagogique. Il a été réalisé sous la direction de Jérôme DELATOUR, en collaboration avec des enseignants et les étudiants de l'option SE de la promotion Poincaré FISA avec le groupe “Cleaner” composé de : 

\begin{itemize}
    \foreach \n in {1,...,6}{\item \Authors(\n)}
\end{itemize}

Ce document est la propriété de Jérôme DELATOUR du groupe ESEO. En dehors des activités pédagogiques de l'ESEO, ce document ne peut être diffusé ou recopié sans l'autorisation écrite de ses propriétaires. 

\subsection{Définitions, acronymes et abréviations}

\begin{tabularx}{\textwidth}{|X X|}
    \hline
    CdC     & Cahier des Charges fourni par le client \\
    \hline
    Client  & Jérôme DELATOUR \\
    \hline
    CU      & Cas d'Utilisation \\
    \hline
    IHM     & Interface Homme Machine \\
    \hline
    UML     & Notation graphique normalisée \\
    \hline
    SàE     & Système à l'Étude. Il s'agit de l'ensemble des logiciels Télécommande et arlogrithme déplacement \\
    \hline
    WIP     & Work In Progress \\
    \hline

\end{tabularx}

\subsection{Vue d'ensemble}

Ce document de spécification est structuré en trois parties : 

\begin{itemize}
    \item La partie I définit l'objet et la portée du document ainsi qu'une liste des abréviations utilisées dans ce document et les références des documents cités. 
    \item La partie II nommée “Description Général”, a pour objectif de présenter l'environnement et le contexte de Projet ainsi que les fonctionnalités principales attendues pour la “Télécommande” et “le robot” 
    \item La parties III présente en détail les IHM attendue du Projet, les fonctionnalités ainsi que le dictionnaire du domaine. 
\end{itemize}